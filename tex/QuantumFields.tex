\chapter{Quantum Field Theory}



\section{Classical Field Theory}


\subsection{What is a field?}

A field $\phi$ is a quantity (eg. Density, Spin, Charge) defined at every point in a manifold $M$ (spacetime, Minkowski space usually).

\begin{equation*}
  \phi : M \leftarrow S
\end{equation*}
So the field is any function from the space to a Target Space.
\begin{itemize}
  \item Here $S$ can be Scalar Field, where $S = \Re$. It good for modeling Higs Boson, Charge Density, Magnetisation density, etc.
  \item Or it can be a vector field $S = \Re^n$. It's good for modelling Pions, Elecromagnetic fields, etc.
  \item We can also have $S = S^2$, which is the surface of a sphere, this is used for the $\sigma$-model and modelling Quantum Magnets.
\end{itemize}

We will restrict our attention to fields whose Classical Dynamics are obtained by applying Variational Principal applied to an Action Functional (Lagrangian Fields). These encode symetries well.


\subsection{Our Lagrangian Field}

We break our vector field down into several scalar fields, $\Phi_a(x)$; $a = 1,2,3\dots,N$.

Our action functional involves the Lagrange density $\mathcal{L}$ which is a function of 
\begin{equation*}
  S(\Omega) = \int_\Omega \mathcal{L}(\partial_\mu \Phi_a) d^4x\;\;;\;\; d^4x = dx_0 dx_1 dx_2 dx_3.
\end{equation*}
Here $\Omega \in \mu_{1,3}$ is a measurable subset of spacetime, the region where $\mathcal{L}$ is defined.

$\mathcal{L}$ is a function if $\Phi_a$, $\partial_\mu \Phi_a$, $\partial_\mu \partial_\nu \Phi_a$, and so on, but we will only take the first derivative, so:
\begin{equation}
  \mathcal{L} = \mathcal{L}(\Phi_a, \partial_\mu \Phi_a)
\end{equation}
And does not depend on the higher derivatives in space, i.e. $\partial_\mu \partial_\nu \Phi_a$ and so on. NOTE that this is an ASSUMPTION, only due to our needs of nice Lagrangian like equations.

We assume that this functional is stationary under small perturbations, i.e. only the second derivative varies, not till the first derivatives. $\Phi_a(x) \longrightarrow \Phi_a(x) + \delta \Phi_a(x)$. But on the boundary 


\subsection{Extracting Equations of Motion}

The Lagrangian density is essentially a highly compressed representation that contains a lot of equations of motion.
We start by taking a path $\Omega$ and creating a perturbation, and evaluate the change in the Lagrangian.
\begin{eqnarray}
  S[\omega] &=& \int_\omega \mathcal{L}(\Phi, \partial_\mu \Phi) \\
  \delta S &=& \int d^4 x \delta \mathcal{L} \\
           &=& \int d^4x (\frac{\partial \mathcal{L}}{\partial \Phi} \cdot \delta \Phi + \frac{\partial \mathcal{L}}{\partial (\partial_\mu \Phi)} \cdot \delta(\partial_\mu \Phi))
\end{eqnarray}
Now we can integrate by parts, first we have the observation on the second term, we see that it's one of the two terms of a product derivative, so we go for the integral. Then we note that our fields die out on the boundaries of space and time, so the integral falls out to 0, and all we are left with is the generalized Euler-Lagrange equations.
\begin{eqnarray}
  \partial_\mu (\frac{\partial \mathcal{L}}{\partial \partial_\mu \Phi} \cdot \delta \Phi) = \partial_\mu (\frac{\partial \mathcal{L}}{\partial \partial_\mu \Phi} \cdot \delta \Phi) + \partial_\mu (\frac{\partial \mathcal{L}}{\partial \partial_\mu \Phi}) \cdot \delta \Phi
\end{eqnarray}


\begin{equation}
  \frac{\partial \mathcal{L}}{\partial \Phi_a} - \frac{\partial}{\partial x^\mu} \frac{\partial \mathcal{L}}{\partial (\partial_\mu \Phi_a)} = 0
\end{equation}


\subsection{Example: Klein Gordon Field}

This is one of the most general examples of a classical field that is Relativistically invarient.
\begin{equation}
  \mathcal{L} = \frac{1}{2} (\partial_0 \Phi)^2 - \frac{1}{2} (\nabla \Phi)^2 - \frac{1}{2} \mu^2 \Phi^2
\end{equation}

Using Euler-Lagrange on this, we get the following equation of motion:
\begin{equation*}
  \frac{\partial}{\partial x^u} (\partial^\mu \Phi) + m^2 \Phi = 0
\end{equation*}
\begin{equation}
  \Box \Phi + m^2 \Phi = 0
\end{equation}
We note that this is quite similar to the Wave Equation, or the Schrodinger equation, but here in 4-D space.


\subsection{Hamiltonian Formalism}

To guess quantum theories with classical limits determined by $\mathcal{L}$, we want to find the conjugate variables, and then impose canonical commutation relations.
If $\Phi_a(x)$ is "canonical position", then the "conjugate momentum density" is $\pi_a(x) = \frac{\partial \mathcal{L}}{\partial \dot{\Phi_a}}$.



\section{Symetries in Classical Field Theory}


\subsection{What are Symetries}
If $\mathcal{L}(\Phi_a, \partial_\mu \Phi_a)$ is a Lagrange density for some field $\Phi_a(x)$, we consider an infinitesimal transformation:
\begin{equation}
  \Phi_a^\prime(x) = \Phi_a(x) + X_a(\Phi_a)
\end{equation}
Then we have a symmetry if and only if
\begin{equation}
  \mathcal{L} \longrightarrow \mathcal{L}(\Phi_a^\prime, \partial_\mu \Phi_a^\prime) = \mathcal{L}(\Phi_a, \partial_\mu \Phi_a) + \partial_\mu F^\mu
\end{equation}
Since this implies that the equations of motion do not change, as the Euler Largrange density is invarient upto a total derivative in the Lagrangian.

\subsection{Noether's Theorem}
Every continuous symmetry of a Lagrangian implies that the existance of a conserved current $j^\mu(x)$, where:
\begin{equation}
  \partial_\mu j^\mu = 0
\end{equation}

\begin{proof}{Noether's Theorem}
  We add a $\delta \Phi_a$ as an infinitesimal arbitrary change.
  \begin{eqnarray*}
    \delta \mathcal{L}(\Phi_a, \partial_\mu \Phi_a) &=& \mathcal{L}(\Phi_a + \delta \Phi_a, \partial_\mu (\Phi_a + \delta \Phi_a)) - \mathcal{L}(\Phi_a, \partial_\mu \Phi_a) \\
    &=& \frac{\partial \mathcal{L}}{\partial \Phi_a} \delta \Phi_a + \frac{\partial \mathcal{L}}{\partial \partial_\mu \Phi_a} \delta \partial_\mu \Phi_a
  \end{eqnarray*}
  \danger
  But we also know, for conservation laws,
  \begin{eqnarray*}
    \delta \mathcal{L} = \partial_\mu F^\mu \\
    \partial_\mu (\frac{\partial \mathcal{L}}{\partial \partial_\mu \Phi_a} X[\Phi_a] - F^\mu) = 0
  \end{eqnarray*}
And here we have the conserved current that we were looking for, in the brackets, and we see it's derivative evaluates to 0.
\end{proof}

This also defines the law of conserved charge, since current density vanishes fast till $\infty$, and when we take all of space we have the conservation law.

We feel that every classical symmetry should have a corresponding quantum symetries, this is because the quantum theories must exhibit those symetries in the classical symetries. Also, these charges can help us derive those symmetries.


\subsection{Examples}

\subsubsection{Spacetime Translations}
% What is active transformation
We consider the active transformation $x^\mu \rightarrow x^\mu - \epsilon^\mu$.

This gives us 4 symetries, which is 4 conserved currents, which gives us 16 numbers in total. % Why 16 numbers.
We get the Energy-Momentum (Stress-Energy) tensor, each column of which corresponds to a conserved current.







\section{Theory Reading}

We think of Lagrangian as a function of time, $\mathcal{L}(t)$, and so we call the Lagrangian dependent on the space, i.e. defined on every point in space as the Lagrangian Density. When we integrate it over space, we call it the Lagrangian. Similarly for the Hamiltonian and the Hamiltonian density.

\begin{definition}{}
  \begin{equation}
  \phi \in \mathbb{C}^2 (\mathcal{M}, S)
  \end{equation}
  This means that $\phi$ is twice differentiable in Complex numbers, and a function from $\mathcal{M}$, the minkowski space, and S is our target space.
\end{definition}

Also note the following, this is what Covarient and Contravarient derivatives mean, specially as in the Klein Gordon equation.
\begin{equation}
  \partial_\mu \partial^\nu = \eta^{\mu \nu} \partial_\mu \partial^\nu
\end{equation}



\section{Quantizing the Field}


\subsection{Motivation}

We are trying to discover Quantum Theories, loosely defined by a Hilbert Space and a Hamiltonian. Mutiple Quantum Theories can correspond to the same Classical Limit, so Quantizing is an Educated Model. If Hamiltonian $\mathcal{H}$ is only dependent upon coordinates and their squares

\begin{eqnarray}
  (q_j, p_j) &\longrightarrow& (\hat{q_j}, \hat{p_j}) \\
  \delta_{j, k} = \{p_j, q_k\}_{j, k} &\longrightarrow& [p_j, q_k] = i \delta_{j, k} \\
  H &\longrightarrow& \hat{H} = \sum_{k} \frac{p_j^2}{2m} + \frac{m}{2} \sum_{j, k} q_j [Q]_{jk} q_k \\
\end{eqnarray}

\subsection{Solving for Hamiltonian}

We know that there exists a diagonalizing orthogonal matrix O, such that $OO^T = \mathbb{I}$. Let this have our $\omega^2$s as it's eigen values, since it's the energy eigenvalues.
\begin{equation}
  O Q O^T = \begin{bmatrix}
    \omega_1^2 & 0 & 0 & ...\\
    0 & \omega_2^2 & 0 & ... \\
    0 & 0 & \omega_3^2 & ... \\
    ... & ... & ... & ...
  \end{bmatrix}
\end{equation}

We take the following relations and substituite into the hamiltonian to get the hamiltonian in the new coordinates.
\begin{eqnarray}
  \hat{q_j^\prime} &=& \sum_{k=1}^n [O]_{jk} \hat{q_k} \\
  \hat{p_j^\prime} &=& \sum_{k=1}^n [O]_{jk} \hat{q_k}
\end{eqnarray}

We get the following Hamiltonian, we can quanitize it (diagonalize it) by constucting the Creator and Annahiator operators.
\begin{equation}
  \hat{H} = \sum_{j=1}^n \frac{p_j^2}{2m} + \frac{1}{2} m \sum_{l=1}^{n} \omega_l^2 \hat{q_l^\prime}^2
\end{equation}



\section{Commutation Relations}

\begin{equation}
  [\Phi_a(x), \Pi^b(y)] = i \delta^{(3)}(\vec{x} - \vec{y}) \delta^b_a
\end{equation}


